% Generated by Sphinx.
\def\sphinxdocclass{report}
\documentclass[letterpaper,10pt,ngerman]{sphinxmanual}

\usepackage[utf8]{inputenc}
\ifdefined\DeclareUnicodeCharacter
  \DeclareUnicodeCharacter{00A0}{\nobreakspace}
\else\fi
\usepackage{cmap}
\usepackage[T1]{fontenc}
\usepackage{amsmath,amssymb}
\usepackage{babel}
\usepackage{times}
\usepackage[Sonny]{fncychap}
\usepackage{longtable}
\usepackage{sphinx}
\usepackage{multirow}
\usepackage{eqparbox}


\addto\captionsngerman{\renewcommand{\figurename}{Abb. }}
\addto\captionsngerman{\renewcommand{\tablename}{Tab. }}
\SetupFloatingEnvironment{literal-block}{name=Quellcode }

\addto\extrasngerman{\def\pageautorefname{page}}

\setcounter{tocdepth}{3}


\title{SampleProject Documentation}
\date{Mai 08, 2016}
\release{1.0}
\author{Richard P}
\newcommand{\sphinxlogo}{}
\renewcommand{\releasename}{Release}
\makeindex

\makeatletter
\def\PYG@reset{\let\PYG@it=\relax \let\PYG@bf=\relax%
    \let\PYG@ul=\relax \let\PYG@tc=\relax%
    \let\PYG@bc=\relax \let\PYG@ff=\relax}
\def\PYG@tok#1{\csname PYG@tok@#1\endcsname}
\def\PYG@toks#1+{\ifx\relax#1\empty\else%
    \PYG@tok{#1}\expandafter\PYG@toks\fi}
\def\PYG@do#1{\PYG@bc{\PYG@tc{\PYG@ul{%
    \PYG@it{\PYG@bf{\PYG@ff{#1}}}}}}}
\def\PYG#1#2{\PYG@reset\PYG@toks#1+\relax+\PYG@do{#2}}

\expandafter\def\csname PYG@tok@gd\endcsname{\def\PYG@tc##1{\textcolor[rgb]{0.63,0.00,0.00}{##1}}}
\expandafter\def\csname PYG@tok@gu\endcsname{\let\PYG@bf=\textbf\def\PYG@tc##1{\textcolor[rgb]{0.50,0.00,0.50}{##1}}}
\expandafter\def\csname PYG@tok@gt\endcsname{\def\PYG@tc##1{\textcolor[rgb]{0.00,0.27,0.87}{##1}}}
\expandafter\def\csname PYG@tok@gs\endcsname{\let\PYG@bf=\textbf}
\expandafter\def\csname PYG@tok@gr\endcsname{\def\PYG@tc##1{\textcolor[rgb]{1.00,0.00,0.00}{##1}}}
\expandafter\def\csname PYG@tok@cm\endcsname{\let\PYG@it=\textit\def\PYG@tc##1{\textcolor[rgb]{0.25,0.50,0.56}{##1}}}
\expandafter\def\csname PYG@tok@vg\endcsname{\def\PYG@tc##1{\textcolor[rgb]{0.73,0.38,0.84}{##1}}}
\expandafter\def\csname PYG@tok@vi\endcsname{\def\PYG@tc##1{\textcolor[rgb]{0.73,0.38,0.84}{##1}}}
\expandafter\def\csname PYG@tok@mh\endcsname{\def\PYG@tc##1{\textcolor[rgb]{0.13,0.50,0.31}{##1}}}
\expandafter\def\csname PYG@tok@cs\endcsname{\def\PYG@tc##1{\textcolor[rgb]{0.25,0.50,0.56}{##1}}\def\PYG@bc##1{\setlength{\fboxsep}{0pt}\colorbox[rgb]{1.00,0.94,0.94}{\strut ##1}}}
\expandafter\def\csname PYG@tok@ge\endcsname{\let\PYG@it=\textit}
\expandafter\def\csname PYG@tok@vc\endcsname{\def\PYG@tc##1{\textcolor[rgb]{0.73,0.38,0.84}{##1}}}
\expandafter\def\csname PYG@tok@il\endcsname{\def\PYG@tc##1{\textcolor[rgb]{0.13,0.50,0.31}{##1}}}
\expandafter\def\csname PYG@tok@go\endcsname{\def\PYG@tc##1{\textcolor[rgb]{0.20,0.20,0.20}{##1}}}
\expandafter\def\csname PYG@tok@cp\endcsname{\def\PYG@tc##1{\textcolor[rgb]{0.00,0.44,0.13}{##1}}}
\expandafter\def\csname PYG@tok@gi\endcsname{\def\PYG@tc##1{\textcolor[rgb]{0.00,0.63,0.00}{##1}}}
\expandafter\def\csname PYG@tok@gh\endcsname{\let\PYG@bf=\textbf\def\PYG@tc##1{\textcolor[rgb]{0.00,0.00,0.50}{##1}}}
\expandafter\def\csname PYG@tok@ni\endcsname{\let\PYG@bf=\textbf\def\PYG@tc##1{\textcolor[rgb]{0.84,0.33,0.22}{##1}}}
\expandafter\def\csname PYG@tok@nl\endcsname{\let\PYG@bf=\textbf\def\PYG@tc##1{\textcolor[rgb]{0.00,0.13,0.44}{##1}}}
\expandafter\def\csname PYG@tok@nn\endcsname{\let\PYG@bf=\textbf\def\PYG@tc##1{\textcolor[rgb]{0.05,0.52,0.71}{##1}}}
\expandafter\def\csname PYG@tok@no\endcsname{\def\PYG@tc##1{\textcolor[rgb]{0.38,0.68,0.84}{##1}}}
\expandafter\def\csname PYG@tok@na\endcsname{\def\PYG@tc##1{\textcolor[rgb]{0.25,0.44,0.63}{##1}}}
\expandafter\def\csname PYG@tok@nb\endcsname{\def\PYG@tc##1{\textcolor[rgb]{0.00,0.44,0.13}{##1}}}
\expandafter\def\csname PYG@tok@nc\endcsname{\let\PYG@bf=\textbf\def\PYG@tc##1{\textcolor[rgb]{0.05,0.52,0.71}{##1}}}
\expandafter\def\csname PYG@tok@nd\endcsname{\let\PYG@bf=\textbf\def\PYG@tc##1{\textcolor[rgb]{0.33,0.33,0.33}{##1}}}
\expandafter\def\csname PYG@tok@ne\endcsname{\def\PYG@tc##1{\textcolor[rgb]{0.00,0.44,0.13}{##1}}}
\expandafter\def\csname PYG@tok@nf\endcsname{\def\PYG@tc##1{\textcolor[rgb]{0.02,0.16,0.49}{##1}}}
\expandafter\def\csname PYG@tok@si\endcsname{\let\PYG@it=\textit\def\PYG@tc##1{\textcolor[rgb]{0.44,0.63,0.82}{##1}}}
\expandafter\def\csname PYG@tok@s2\endcsname{\def\PYG@tc##1{\textcolor[rgb]{0.25,0.44,0.63}{##1}}}
\expandafter\def\csname PYG@tok@nt\endcsname{\let\PYG@bf=\textbf\def\PYG@tc##1{\textcolor[rgb]{0.02,0.16,0.45}{##1}}}
\expandafter\def\csname PYG@tok@nv\endcsname{\def\PYG@tc##1{\textcolor[rgb]{0.73,0.38,0.84}{##1}}}
\expandafter\def\csname PYG@tok@s1\endcsname{\def\PYG@tc##1{\textcolor[rgb]{0.25,0.44,0.63}{##1}}}
\expandafter\def\csname PYG@tok@ch\endcsname{\let\PYG@it=\textit\def\PYG@tc##1{\textcolor[rgb]{0.25,0.50,0.56}{##1}}}
\expandafter\def\csname PYG@tok@m\endcsname{\def\PYG@tc##1{\textcolor[rgb]{0.13,0.50,0.31}{##1}}}
\expandafter\def\csname PYG@tok@gp\endcsname{\let\PYG@bf=\textbf\def\PYG@tc##1{\textcolor[rgb]{0.78,0.36,0.04}{##1}}}
\expandafter\def\csname PYG@tok@sh\endcsname{\def\PYG@tc##1{\textcolor[rgb]{0.25,0.44,0.63}{##1}}}
\expandafter\def\csname PYG@tok@ow\endcsname{\let\PYG@bf=\textbf\def\PYG@tc##1{\textcolor[rgb]{0.00,0.44,0.13}{##1}}}
\expandafter\def\csname PYG@tok@sx\endcsname{\def\PYG@tc##1{\textcolor[rgb]{0.78,0.36,0.04}{##1}}}
\expandafter\def\csname PYG@tok@bp\endcsname{\def\PYG@tc##1{\textcolor[rgb]{0.00,0.44,0.13}{##1}}}
\expandafter\def\csname PYG@tok@c1\endcsname{\let\PYG@it=\textit\def\PYG@tc##1{\textcolor[rgb]{0.25,0.50,0.56}{##1}}}
\expandafter\def\csname PYG@tok@o\endcsname{\def\PYG@tc##1{\textcolor[rgb]{0.40,0.40,0.40}{##1}}}
\expandafter\def\csname PYG@tok@kc\endcsname{\let\PYG@bf=\textbf\def\PYG@tc##1{\textcolor[rgb]{0.00,0.44,0.13}{##1}}}
\expandafter\def\csname PYG@tok@c\endcsname{\let\PYG@it=\textit\def\PYG@tc##1{\textcolor[rgb]{0.25,0.50,0.56}{##1}}}
\expandafter\def\csname PYG@tok@mf\endcsname{\def\PYG@tc##1{\textcolor[rgb]{0.13,0.50,0.31}{##1}}}
\expandafter\def\csname PYG@tok@err\endcsname{\def\PYG@bc##1{\setlength{\fboxsep}{0pt}\fcolorbox[rgb]{1.00,0.00,0.00}{1,1,1}{\strut ##1}}}
\expandafter\def\csname PYG@tok@mb\endcsname{\def\PYG@tc##1{\textcolor[rgb]{0.13,0.50,0.31}{##1}}}
\expandafter\def\csname PYG@tok@ss\endcsname{\def\PYG@tc##1{\textcolor[rgb]{0.32,0.47,0.09}{##1}}}
\expandafter\def\csname PYG@tok@sr\endcsname{\def\PYG@tc##1{\textcolor[rgb]{0.14,0.33,0.53}{##1}}}
\expandafter\def\csname PYG@tok@mo\endcsname{\def\PYG@tc##1{\textcolor[rgb]{0.13,0.50,0.31}{##1}}}
\expandafter\def\csname PYG@tok@kd\endcsname{\let\PYG@bf=\textbf\def\PYG@tc##1{\textcolor[rgb]{0.00,0.44,0.13}{##1}}}
\expandafter\def\csname PYG@tok@mi\endcsname{\def\PYG@tc##1{\textcolor[rgb]{0.13,0.50,0.31}{##1}}}
\expandafter\def\csname PYG@tok@kn\endcsname{\let\PYG@bf=\textbf\def\PYG@tc##1{\textcolor[rgb]{0.00,0.44,0.13}{##1}}}
\expandafter\def\csname PYG@tok@cpf\endcsname{\let\PYG@it=\textit\def\PYG@tc##1{\textcolor[rgb]{0.25,0.50,0.56}{##1}}}
\expandafter\def\csname PYG@tok@kr\endcsname{\let\PYG@bf=\textbf\def\PYG@tc##1{\textcolor[rgb]{0.00,0.44,0.13}{##1}}}
\expandafter\def\csname PYG@tok@s\endcsname{\def\PYG@tc##1{\textcolor[rgb]{0.25,0.44,0.63}{##1}}}
\expandafter\def\csname PYG@tok@kp\endcsname{\def\PYG@tc##1{\textcolor[rgb]{0.00,0.44,0.13}{##1}}}
\expandafter\def\csname PYG@tok@w\endcsname{\def\PYG@tc##1{\textcolor[rgb]{0.73,0.73,0.73}{##1}}}
\expandafter\def\csname PYG@tok@kt\endcsname{\def\PYG@tc##1{\textcolor[rgb]{0.56,0.13,0.00}{##1}}}
\expandafter\def\csname PYG@tok@sc\endcsname{\def\PYG@tc##1{\textcolor[rgb]{0.25,0.44,0.63}{##1}}}
\expandafter\def\csname PYG@tok@sb\endcsname{\def\PYG@tc##1{\textcolor[rgb]{0.25,0.44,0.63}{##1}}}
\expandafter\def\csname PYG@tok@k\endcsname{\let\PYG@bf=\textbf\def\PYG@tc##1{\textcolor[rgb]{0.00,0.44,0.13}{##1}}}
\expandafter\def\csname PYG@tok@se\endcsname{\let\PYG@bf=\textbf\def\PYG@tc##1{\textcolor[rgb]{0.25,0.44,0.63}{##1}}}
\expandafter\def\csname PYG@tok@sd\endcsname{\let\PYG@it=\textit\def\PYG@tc##1{\textcolor[rgb]{0.25,0.44,0.63}{##1}}}

\def\PYGZbs{\char`\\}
\def\PYGZus{\char`\_}
\def\PYGZob{\char`\{}
\def\PYGZcb{\char`\}}
\def\PYGZca{\char`\^}
\def\PYGZam{\char`\&}
\def\PYGZlt{\char`\<}
\def\PYGZgt{\char`\>}
\def\PYGZsh{\char`\#}
\def\PYGZpc{\char`\%}
\def\PYGZdl{\char`\$}
\def\PYGZhy{\char`\-}
\def\PYGZsq{\char`\'}
\def\PYGZdq{\char`\"}
\def\PYGZti{\char`\~}
% for compatibility with earlier versions
\def\PYGZat{@}
\def\PYGZlb{[}
\def\PYGZrb{]}
\makeatother

\renewcommand\PYGZsq{\textquotesingle}

\begin{document}
\shorthandoff{"}
\maketitle
\tableofcontents
\phantomsection\label{index::doc}



\chapter{SampleProject}
\label{index:sampleproject}\label{index:willkommen-bei-der-testdoku}

\section{Main module}
\label{Main:module-Main}\label{Main:main-module}\label{Main::doc}\index{Main (Modul)}
Contains a BaseClass, an Interface and a ChildClass, implementing the interface
\index{MyBaseClass (Klasse in Main)}

\begin{fulllineitems}
\phantomsection\label{Main:Main.MyBaseClass}\pysigline{\strong{class }\code{Main.}\bfcode{MyBaseClass}}
Basisklassen: \code{object}

The BaseClass
\index{BaseClassMember (Attribut von Main.MyBaseClass)}

\begin{fulllineitems}
\phantomsection\label{Main:Main.MyBaseClass.BaseClassMember}\pysigline{\bfcode{BaseClassMember}\strong{ = `Hello'}}
This is a simple text member of the BaseClass

\end{fulllineitems}

\index{set\_baseclassmember() (Methode von Main.MyBaseClass)}

\begin{fulllineitems}
\phantomsection\label{Main:Main.MyBaseClass.set_baseclassmember}\pysiglinewithargsret{\bfcode{set\_baseclassmember}}{\emph{text}}{}
Sets self.BaseClassMember to text.
\begin{description}
\item[{Args:}] \leavevmode
text (str): The text to set BaseClassMember to.

\end{description}

\end{fulllineitems}


\end{fulllineitems}

\index{MyClass (Klasse in Main)}

\begin{fulllineitems}
\phantomsection\label{Main:Main.MyClass}\pysigline{\strong{class }\code{Main.}\bfcode{MyClass}}
Basisklassen: {\hyperref[Main:Main.MyInterface]{\crossref{\code{Main.MyInterface}}}}, {\hyperref[Main:Main.MyBaseClass]{\crossref{\code{Main.MyBaseClass}}}}

The actual Class
\index{SomeMember (Attribut von Main.MyClass)}

\begin{fulllineitems}
\phantomsection\label{Main:Main.MyClass.SomeMember}\pysigline{\bfcode{SomeMember}\strong{ = 5}}
This is a simple int member of the MyClass

\end{fulllineitems}

\index{set\_somemember() (Methode von Main.MyClass)}

\begin{fulllineitems}
\phantomsection\label{Main:Main.MyClass.set_somemember}\pysiglinewithargsret{\bfcode{set\_somemember}}{\emph{number}}{}
Sets self.SomeMember to number.
\begin{description}
\item[{Args:}] \leavevmode
number (int): The number to set self.SomeMember to.

\end{description}

Use it like this:

\begin{Verbatim}[commandchars=\\\{\}]
\PYG{g+gp}{\PYGZgt{}\PYGZgt{}\PYGZgt{} }\PYG{n}{MyClass}\PYG{p}{(}\PYG{p}{)}\PYG{o}{.}\PYG{n}{set\PYGZus{}somemember}\PYG{p}{(}\PYG{l+m+mi}{5}\PYG{p}{)}
\end{Verbatim}

\end{fulllineitems}


\end{fulllineitems}

\index{MyInterface (Klasse in Main)}

\begin{fulllineitems}
\phantomsection\label{Main:Main.MyInterface}\pysigline{\strong{class }\code{Main.}\bfcode{MyInterface}}
Basisklassen: \code{object}

The Interface
\index{Interfacemethod() (Methode von Main.MyInterface)}

\begin{fulllineitems}
\phantomsection\label{Main:Main.MyInterface.Interfacemethod}\pysiglinewithargsret{\bfcode{Interfacemethod}}{}{}
A method of the Interface

\end{fulllineitems}


\end{fulllineitems}



\section{Util package}
\label{Util::doc}\label{Util:util-package}

\subsection{Submodules}
\label{Util:submodules}

\subsection{Util.Util module}
\label{Util:module-Util.Util}\label{Util:util-util-module}\index{Util.Util (Modul)}\index{UtilClass (Klasse in Util.Util)}

\begin{fulllineitems}
\phantomsection\label{Util:Util.Util.UtilClass}\pysigline{\strong{class }\code{Util.Util.}\bfcode{UtilClass}}
Basisklassen: \code{object}

An utilityclass.
\index{DoSomething() (Methode von Util.Util.UtilClass)}

\begin{fulllineitems}
\phantomsection\label{Util:Util.Util.UtilClass.DoSomething}\pysiglinewithargsret{\bfcode{DoSomething}}{}{}
Does something to itself.

\end{fulllineitems}


\end{fulllineitems}

\index{UtilMethod1() (im Modul Util.Util)}

\begin{fulllineitems}
\phantomsection\label{Util:Util.Util.UtilMethod1}\pysiglinewithargsret{\code{Util.Util.}\bfcode{UtilMethod1}}{\emph{argument}}{}
This Utilmethod takes one argument.

\end{fulllineitems}

\index{UtilMethod2() (im Modul Util.Util)}

\begin{fulllineitems}
\phantomsection\label{Util:Util.Util.UtilMethod2}\pysiglinewithargsret{\code{Util.Util.}\bfcode{UtilMethod2}}{\emph{argument1}, \emph{argument2}}{}
This Utilmethod takes two arguments.

\end{fulllineitems}



\subsection{Module contents}
\label{Util:module-Util}\label{Util:module-contents}\index{Util (Modul)}
Contains only the Util file...
\index{start() (im Modul Util)}

\begin{fulllineitems}
\phantomsection\label{Util:Util.start}\pysiglinewithargsret{\code{Util.}\bfcode{start}}{}{}
This is just some Module-Level Method.

\end{fulllineitems}


Inhalt:


\renewcommand{\indexname}{Python-Modulindex}
\begin{theindex}
\def\bigletter#1{{\Large\sffamily#1}\nopagebreak\vspace{1mm}}
\bigletter{m}
\item {\texttt{Main}}, \pageref{Main:module-Main}
\indexspace
\bigletter{u}
\item {\texttt{Util}}, \pageref{Util:module-Util}
\item {\texttt{Util.Util}}, \pageref{Util:module-Util.Util}
\end{theindex}

\renewcommand{\indexname}{Stichwortverzeichnis}
\printindex
\end{document}
